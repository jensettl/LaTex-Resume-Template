\documentclass{resume} % Use the custom resume.cls style

\usepackage[left=0.4 in,top=0.4in,right=0.4 in,bottom=0.4in]{geometry} % Document margins
\usepackage{changepage}
\usepackage{fontawesome}
\usepackage{ragged2e}
\usepackage[english,ngerman]{babel}

\begin{document}

%----------------------------------------------------------------------------------------
%	WORK EXPERIENCE SECTION
%----------------------------------------------------------------------------------------

\begin{rSection}{Projekte}
    \vspace{0.5em}
    \justifying{In meinem Studium und meiner Freizeit  habe ich sowohl eigenständig als auch im Team eine Vielzahl von Projekten durchgeführt, 
    die ein breites Spektrum an Themen umfassen.
    Diese reichen von der Planung und Konzeption mithilfe von Office-Produkten wie Excel und Word 
    bis hin zur Anwendung von SCRUM-Methoden für die Entwicklung und Durchführung:}

    
    \begin{adjustwidth}{0.5cm}{}
        \vspace{-1.5em}

        \item \textbf{Lebenslauf Analyst mit AI} \hfill {Januar 2024} \\
        \textit{(Python, Ollama, Streamlit)} \hfill {1 Person}\\

        \item \textbf{MailService auf Raspberry für als Reporting System} \hfill {Februar 2024} \\
        \textit{(Python, Hosting)} \hfill {1 Person}\\
        {Konzeption und Implementierung eines automatisierten Reporting-Systems, das wöchentlich umfassende Daten zu verschiedenen Lebensbereichen wie Wetter und Aktienkurse generiert und diese per Email an mich sendet.}
        

        \item \textbf{N8N Workflow Automatisierung mit AI Integration} \hfill {Januar 2024} \\
        \textit{(N8N, OpenAI API)} \hfill {1 Person}\\
        \raggedright{Entwicklung eines N8N-Workflows zur Automatisierung von Aufgaben, der die OpenAI API integriert.
        Der Workflow wurde so konzipiert, dass er Daten von verschiedenen Quellen abruft, diese mithilfe der OpenAI API analysiert und die Ergebnisse in einem strukturierten Format speichert.
        Die OpenAI API wird verwendet, um die Daten zu analysieren und relevante Informationen zu extrahieren.
        Der Workflow ist in N8N implementiert und nutzt verschiedene Integrationen, um Daten von verschiedenen Quellen abzurufen und die Ergebnisse zu speichern.
        Die Ergebnisse werden in einem strukturierten Format gespeichert, das leicht in andere Anwendungen importiert werden kann.}

        \item \textbf{Web Scraper mit AI Enhancements} \hfill {Januar 2024} \\
        \textit{(Python, BeautifulSoup, OpenAI API)} \hfill {1 Person}\\
        \raggedright{Entwicklung eines Web-Scrapers, der Daten von einer Website extrahiert und diese mithilfe der OpenAI API analysiert.
        Der Scraper wurde so konzipiert, dass er die Daten in einem strukturierten Format speichert und die Ergebnisse in einer CSV-Datei ausgibt.
        Die OpenAI API wird verwendet, um die extrahierten Daten zu analysieren und relevante Informationen zu extrahieren.
        Der Scraper ist in Python geschrieben und verwendet die BeautifulSoup-Bibliothek, um die HTML-Seiten zu parsen und die gewünschten Daten zu extrahieren.
        Die Ergebnisse werden in einer CSV-Datei gespeichert, die leicht in andere Anwendungen importiert werden kann.}

        \item \textbf{Portfolio Website} \hfill {Januar 2024} \\
        \textit{(React, Tailwind, Firebase Hosting)} \hfill {1 Person}\\ 
        \raggedright{Entwurf, Entwicklung und Verwaltung meiner persönlichen Online-Visitenkarte zur Präsentation meiner Projekte, Fähigkeiten und
        Erfahrungen. \href{https://www.jensettl.com//}{(Online verfügbar unter www.jensettl.com)}}

        \item \textbf{Identifizierung fehlerhafter Produkte} \hfill {März 2023} \\
        \textit{(Machine Learning, Pandas, Python)} \hfill {1 Person}\\
        \justifying{Aufbereitung und Analyse eines Datensatzes aus der Stahlproduktion mit Schwerpunkt auf Oberflächenunregel-mäßigkeiten.
        Implementierung datengesteuerter Lösungen zur frühzeitigen Erkennung fehlerhafter Stahlprodukte
        unter Anwendung von Algorithmen des maschinellen Lernens.}
    
        \item \textbf{Supply Chain Management Simulation} \hfill {Dezember 2022 - März 2023} \\
        \textit{(Typescript, FastAPI, SCM, Excel)} \hfill {4 Personen}\\ 
        \justifying{Durchführung eines Planspiels, um eine Lieferkette in der Fahrradindustrie zu simulieren. Hierfür wurde Microsoft Excel genutzt.
        Anschließende Entwicklung von Frontend und Backend mit Typescript und FastAPI um einen Planungszeitraum zu berechnen und zu simulieren. 
        Die Ergebnisse wurden in einer Abschlusspräsentation vorgestellt.}
    
        \item \textbf{Anwendungsprojekt SmartCampus} \hfill {April 2022 - August 2022} \\
        \textit{(Planung, Konzeption, Entwicklung)} \hfill {4 Personen} \\
        \justifying{In einem realen Kundenprojekt auf dem Hochschulcampus führte wir als Gruppe die Evaluation 
        sensorbasierter Mehrwertdienste durch. Ein regelmäßiger Austausch mit dem Kunden war dabei essenziell, 
        um die Anforderungen zu verstehen und sicherzustellen, dass die Budgetplanung eingehalten wurde. 
        Durch das Projekt waren wir maßgeblich an der Planung, Konzeption und Durchführung beteiligt. 
        Insbesondere gehörte es zu unseren Aufgaben, die Ergebnisse der Gespräche mit dem Kunden in konkrete Aufgaben 
        umzusetzen und an das Team zu delegieren, um eine effiziente Umsetzung zu gewährleisten.}
        
        \item \textbf{Automatisierung eines Kundenauftragsprozesses} \hfill {Februar 2022} \\
        \textit{(BPMN, Java, Camunda)} \hfill {4 Personen}\\
        \justifying{{Abbilden eines Kundenauftragsprozesses in BPMN und Untersuchung auf Automatisierungsmöglichkeiten.
        Anschließende Implementierung der Automatisierungsentwürfe als Gruppe und Simulation mit Camunda.}}
    
        \item \textbf{gRPC Client-Server Kommunikation} \hfill {Oktober 2021} \\
        \textit{(JavaScript, gRPC)} \hfill {März 2023}\\
        {Entwicklung einer effizienten Client-Server Kommunikation mit gRPC im Rahmen der Veranstaltung “Automatisierung von Geschäftsprozessen”.
        Dokumentation und Code wurde mit GitHub verwaltet. Aufgaben wurden mit Kanban in Trello organisiert.}
        
        \item \textbf{Git Workshop} \hfill {September 2021} \\
        \textit{(Git, Markdown)} \hfill {3 Personen}\\
        {Entwicklung eines Git-Workshops für Studenten zum Erlernen der Grundlagen von Git und GitHub.
        Der Workshop fand an der Hochschule für Technik und Wirtschaft Karlsruhe statt.}
    
        \item \textbf{Untersuchung von Unternehmensdaten auf Compliance} \hfill {Mai 2021} \\
        \textit{(Datenanalyse, JMP)} \hfill {2 Personen} \\
        {Untersuchung realer Unternehmensdaten mit Hilfe von statistischen Methoden, Entscheidungsbäumen und neuronalen Netzen,
        welche Arbeitsschritte in einem Produktionsprozess die Einhaltung der Vorschriften beeinflussen.}
   
        \item \textbf{TODO Automatisiertes digitales Aufräumsystem} \hfill {Dezember 2023}\\ 
        \textit{(Python, CRON)} \hfill {1 Person}\\
        {Automatisiertes Sortieren und Verschieben von Dateien in verschiedene Ordner basierend auf Dateityp und Erstellungsdatum.}
        
        \item \textbf{HEIC-JPEG Converter} \hfill {Dezember 2023}\\
        \textit{(Python)} \hfill {1 Person}\\
        {Problestellung: HEIC Dateien können nicht auf allen Geräten geöffnet werden und Online-Konvertierer erfordern den Upload persönlicher Daten \\
        Lösung: Lokale Konvertierung von HEIC Dateien in das weit verbreitete JPEG Format.}
        
        \item \textbf{Modellierung eines IT-Modells} \hfill {März 2020 - Juni 2020} \\
        \textit{(Konzeption, Use-Cases, Persona)} \hfill {3 Personen}\\ 
        {In der Veranstaltung "Modellierung von IT-Systemen" ging es um die Konzeption und Planung einer App mit Fokus auf den Kundenanforderungen, Use-Cases bis hin zur Abbildung der Prozesse mit BPMN und UML.}
        
        \item \textbf{Nett-Hier Schnitzeljagd - Konzept} \hfill {Oktober 2023 - November 2023} \\
        {Ausgestaltung einer Idee, eine Website zu entwicklen, die es ermöglicht, Fotos von Nett-Hier Sticker und deren Standort zu teilen. Als eine Art digitale Schnitzeljagd nach Stickern auf der ganzen Welt.}
    
        \item \textbf{Notion-Datenbank API Automatisierung} \hfill {Oktober 2023 - November 2023} \\
        
        \item \textbf{Event Management Website mit Bolt.new} \hfill {Oktober 2023 - November 2023} \\
        \textit{(HTML, CSS, JavaScript)} \hfill {1 Person}\\

        \item \textbf{Resume Website Creator using Langflow} \hfill {Oktober 2023 - November 2023} \\
        \textit{(Langflow, Python)} \hfill {1 Person}\\
        \raggedright{Generate a single-page portfolio website using HTML and CSS that takes a resume in JSON format as input and dynamically renders the following sections with a well-structured and aesthetic layout.}

        \item \textbf{Inflation Prediction Model} \hfill {Oktober 2023 - November 2023} \\
        \textit{(Python, Pandas, Scikit-learn)} \hfill {1 Person}\\
        \raggedright{Developed a machine learning model to predict inflation rates using historical economic data. The model was trained on various economic indicators and evaluated for accuracy. The results were visualized using Matplotlib and Seaborn.}

    \end{adjustwidth}
    
\end{rSection}

\end{document}