\documentclass{resume} % Use the custom resume.cls style

\usepackage[left=0.4 in,top=0.4in,right=0.4 in,bottom=0.4in]{geometry} % Document margins
\usepackage{changepage}
\usepackage{fontawesome}
\usepackage{ragged2e}
\usepackage[english,ngerman]{babel}
\newcommand{\tab}[1]{\hspace{.2667\textwidth}\rlap{#1}} 
\newcommand{\itab}[1]{\hspace{0em}\rlap{#1}}
\newcommand{\linkedin}{\faLinkedin\hspace{0.2em}}
\newcommand{\github}{\faGithub\hspace{0.2em}}
\name{Jens Ettl} % Your name
\address{+49 176 78867038 \\ Karlsruhe, Deutschland \\ \today} 
\address{\href{mailto:ettljens@gmail.com}{ettljens@gmail.com} \\ \linkedin \href{https://www.linkedin.com/in/jens-ettl-807578211/}{www.linkedin.com/in/jens-ettl} \\ \github \href{https://github.com/xy}{github.com/jensettl} \\ \href{www.jensettl.com}{www.jensettl.com}
} 


\begin{document}

%----------------------------------------------------------------------------------------
%	EDUCATION SECTION
%----------------------------------------------------------------------------------------

\begin{rSection}{Bildungsweg}
    {\bf Bachelor Wirtschaftsinformatik}, Hochschule Karlsruhe - Technik und Wirtschaft \hfill {Erwartet 2024}
    \begin{itemize}
        \item Tutoriumsleitung und -organisation in den Modulen Datenbanken 1 und Statistik u. Business Intelligence.
        \item \raggedright{Wahlmodule: Erfolgreiche Durchführung von Data Science Projekten, Datenanalyse mit JMP, Machine Learning, User Centered Design}\\
    \end{itemize}

    {\bf Bachelor Informatik}, Karlsruhe Institut für Technologie \hfill {\textit{Abgebrochen} 2016 - 2018}

    {\bf Allgemeine Hochschulreife} \textit{Leistungsfächer Informatik, Mathematik und Physik} \hfill {2016}

\end{rSection}

%----------------------------------------------------------------------------------------
% TECHINICAL STRENGTHS	
%----------------------------------------------------------------------------------------
\begin{rSection}{Fähigkeiten und Fertigkeiten}

    \begin{tabular}{ @{} >{\bfseries}l @{\hspace{6ex}} l }
        Technical Skills & Programmiersprachen (Python, JavaScript), UML, BPMN, Adobe XD,                            \\
                        & SQL, Firebase, VisualStudioCode, Docker, Confluence, Jira, Microsoft Office               \\
        \\
        Soft Skills     & Teamwork, Kommunikation, Problemlösung, Strukturiertheit, Pünktlichkeit,                  \\
                        & Kreativität, Schnelle Auffassungsgabe                                                     \\
        \\
        Sonstiges       & Englisch \textit{(Verhandlungssicher)}, Französisch \textit{(vertiefte Grundkenntnisse)}, \\
                        & Führerschein A \& B
    \end{tabular}\\
\end{rSection}

%----------------------------------------------------------------------------------------
% Work Experience
%----------------------------------------------------------------------------------------

\begin{rSection}{Berufserfahrung}

    \textbf{Productowner} \hfill Okt 2021 - März 2023\\
    CAS Software AG \hfill \textit{Karlsruhe, Deutschland}
    \begin{itemize}
        \itemsep -3pt {}
        \item \raggedright{\textbf{Produktentwicklung unterstützen:} Mitwirkung an der Konzeption und Entwicklung von Apps und MVPs, sowie Kenntnisse im Umgang mit CRM- und ERP-Systemen.}
        \item \raggedright{\textbf{Kundenanforderungen umsetzen:} Erfahrung in der direkten Arbeit mit Kundenanforderungen, inklusive Erstellung von User-Stories und Akzeptanzkriterien}
        \item \raggedright{\textbf{Sprint-Planung und -Durchführung:} Verantwortlich für die Planung und Durchführung von Sprint-Reviews und Retrospektiven mit SCRUM.}
    \end{itemize}

    \textbf{Vorstand der Fachschaft Wirtschaftsinformatik} \hfill Okt 2020 - März 2022\\
    Technische Hochschule Karlsruhe \hfill \textit{Karlsruuhe, Deutschland}
    \begin{itemize}
        \itemsep -3pt {}
        \item \raggedright{\textbf{Beziehungen und Interessenvertretung:} Generierung neuer Firmenbeziehungen, Vertretung der Studierendeninteressen gegenüber der Hochschulleitung und Mitwirkung bei politischen Entscheidungen.}
        \item \raggedright{\textbf{Teamentwicklung und Konfliktlösung:} Steigerung der Teamfähigkeit durch die Planung und Durchführung von Events und Workshops sowie das Management von Konflikten zwischen Studierenden und Dozenten.} 
        \item \raggedright{\textbf{Kommunikation und Dokumentation:} Kommunikation mit der Hochschulleitung, Erstellung von Präsentationen und Dokumentationen zur Unterstützung der Fachschaftsarbeit.}
    \end{itemize}
\end{rSection}

%----------------------------------------------------------------------------------------
% Additional Activities
%----------------------------------------------------------------------------------------

\begin{rSection}{Zusätzliche Aktivitäten}
    \begin{itemize}
        \item 	Teilnahme an Workshops zu SCRUM und Qualitätsmanagement.
        \item	Mentor an der Hochschule für Erstisemestler und Quereinsteiger.
        \item   Sprachaufenthalt in England 2014
    \end{itemize}


\end{rSection}

%----------------------------------------------------------------------------------------
%	WORK EXPERIENCE SECTION
%----------------------------------------------------------------------------------------

\newpage
\begin{rSection}{Projekte}
    \vspace{0.5em}
    \justifying{In meinem Studium und meiner Freizeit  habe ich sowohl eigenständig als auch im Team eine Vielzahl von Projekten durchgeführt, 
    die ein breites Spektrum an Themen umfassen.
    Diese reichen von der Planung und Konzeption mithilfe von Office-Produkten wie Excel und Word 
    bis hin zur Anwendung von SCRUM-Methoden für die Entwicklung und Durchführung:}

    
    \begin{adjustwidth}{0.5cm}{}
        \vspace{-1.5em}
        \item \textbf{Portfolio Website} \hfill {Januar 2024} \\
        \textit{(React, Tailwind, Firebase Hosting)} \hfill {1 Person}\\ 
        \raggedright{Entwurf, Entwicklung und Verwaltung meiner persönlichen Online-Visitenkarte zur Präsentation meiner Projekte, Fähigkeiten und
        Erfahrungen. \href{https://www.jensettl.com//}{(Online verfügbar unter www.jensettl.com)}}

        \item \textbf{Identifizierung fehlerhafter Produkte} \hfill {März 2023} \\
        \textit{(Machine Learning, Pandas, Python)} \hfill {1 Person}\\
        \justifying{Aufbereitung und Analyse eines Datensatzes aus der Stahlproduktion mit Schwerpunkt auf Oberflächenunregel-mäßigkeiten.
        Implementierung datengesteuerter Lösungen zur frühzeitigen Erkennung fehlerhafter Stahlprodukte
        unter Anwendung von Algorithmen des maschinellen Lernens.}
    
        \item \textbf{Supply Chain Management Simulation} \hfill {Dezember 2022 - März 2023} \\
        \textit{(Typescript, FastAPI, SCM, Excel)} \hfill {4 Personen}\\ 
        \justifying{Durchführung eines Planspiels, um eine Lieferkette in der Fahrradindustrie zu simulieren. Hierfür wurde Microsoft Excel genutzt.
        Anschließende Entwicklung von Frontend und Backend mit Typescript und FastAPI um einen Planungszeitraum zu berechnen und zu simulieren. 
        Die Ergebnisse wurden in einer Abschlusspräsentation vorgestellt.}
    
        \item \textbf{Anwendungsprojekt SmartCampus} \hfill {April 2022 - August 2022} \\
        \textit{(Planung, Konzeption, Entwicklung)} \hfill {4 Personen} \\
        \justifying{In einem realen Kundenprojekt auf dem Hochschulcampus führte wir als Gruppe die Evaluation 
        sensorbasierter Mehrwertdienste durch. Ein regelmäßiger Austausch mit dem Kunden war dabei essenziell, 
        um die Anforderungen zu verstehen und sicherzustellen, dass die Budgetplanung eingehalten wurde. 
        Durch das Projekt waren wir maßgeblich an der Planung, Konzeption und Durchführung beteiligt. 
        Insbesondere gehörte es zu unseren Aufgaben, die Ergebnisse der Gespräche mit dem Kunden in konkrete Aufgaben 
        umzusetzen und an das Team zu delegieren, um eine effiziente Umsetzung zu gewährleisten.}
        
        \item \textbf{Automatisierung eines Kundenauftragsprozesses} \hfill {Februar 2022} \\
        \textit{(BPMN, Java, Camunda)} \hfill {4 Personen}\\
        \justifying{{Abbilden eines Kundenauftragsprozesses in BPMN und Untersuchung auf Automatisierungsmöglichkeiten.
        Anschließende Implementierung der Automatisierungsentwürfe als Gruppe und Simulation mit Camunda.}}
    
        \item \textbf{gRPC Client-Server Kommunikation} \hfill {Oktober 2021} \\
        \textit{(JavaScript, gRPC)} \hfill {März 2023}\\
        {Entwicklung einer effizienten Client-Server Kommunikation mit gRPC im Rahmen der Veranstaltung “Automatisierung von Geschäftsprozessen”.
        Dokumentation und Code wurde mit GitHub verwaltet. Aufgaben wurden mit Kanban in Trello organisiert.}
        
        \item \textbf{Git Workshop} \hfill {September 2021} \\
        \textit{(Git, Markdown)} \hfill {3 Personen}\\
        {Entwicklung eines Git-Workshops für Studenten zum Erlernen der Grundlagen von Git und GitHub.
        Der Workshop fand an der Hochschule für Technik und Wirtschaft Karlsruhe statt.}
    
        \item \textbf{Untersuchung von Unternehmensdaten auf Compliance} \hfill {Mai 2021} \\
        \textit{(Datenanalyse, JMP)} \hfill {2 Personen} \\
        {Untersuchung realer Unternehmensdaten mit Hilfe von statistischen Methoden, Entscheidungsbäumen und neuronalen Netzen,
        welche Arbeitsschritte in einem Produktionsprozess die Einhaltung der Vorschriften beeinflussen.}
    
        \item \textbf{Automatisches Reporting System} \hfill {Februar 2024} \\
        \textit{(Python, Hosting)} \hfill {1 Person}\\
        {Konzeption und Implementierung eines automatisierten Reporting-Systems, das wöchentlich umfassende Daten zu verschiedenen Lebensbereichen wie Wetter und Aktienkurse generiert und diese per Email an mich sendet.}
        
        \newpage 
        \item \textbf{Automatisiertes digitales Aufräumsystem} \hfill {Dezember 2023}\\ 
        \textit{(Python, CRON)} \hfill {1 Person}\\
        {Automatisiertes Sortieren und Verschieben von Dateien in verschiedene Ordner basierend auf Dateityp und Erstellungsdatum.}
        
        \item \textbf{HEIC-JPEG Converter} \hfill {Dezember 2023}\\
        \textit{(Python)} \hfill {1 Person}\\
        {Problestellung: HEIC Dateien können nicht auf allen Geräten geöffnet werden und Online-Konvertierer erfordern den Upload persönlicher Daten \\
        Lösung: Lokale Konvertierung von HEIC Dateien in das weit verbreitete JPEG Format.}
        
        \item \textbf{Modellierung eines IT-Modells} \hfill {März 2020 - Juni 2020} \\
        \textit{(Konzeption, Use-Cases, Persona)} \hfill {3 Personen}\\ 
        {In der Veranstaltung "Modellierung von IT-Systemen" ging es um die Konzeption und Planung einer App mit Fokus auf den Kundenanforderungen, Use-Cases bis hin zur Abbildung der Prozesse mit BPMN und UML.}
        
        \item \textbf{Nett-Hier Schnitzeljagd - Konzept} \hfill {Oktober 2023 - November 2023} \\
        {Ausgestaltung einer Idee, eine Website zu entwicklen, die es ermöglicht, Fotos von Nett-Hier Sticker und deren Standort zu teilen. Als eine Art digitale Schnitzeljagd nach Stickern auf der ganzen Welt.}
    
    \end{adjustwidth}
    
\end{rSection}


%----------------------------------------------------------------------------------------
% Achievements and Certificates
%----------------------------------------------------------------------------------------

\begin{rSection}{Auszeichnungen und Zertifikate}
    \vspace{0.5em}

    \begin{itemize}
        \item \textbf{Arbeitszeugnis}, CAS Software AG
        \item \textbf{Modern JavaScript}, Udemy
        \item \textbf{Grundlagen des Onlinemarketings}, Google Zukunftswerkstatt
        \item \textbf{RPA Starter}, UI Path
    \end{itemize}


\end{rSection}

%----------------------------------------------------------------------------------------
% Hobbies and Interests
%----------------------------------------------------------------------------------------

\begin{rSection}{Freizeit und Hobbies}
    \vspace{0.5em}
    \begin{itemize}
        \item \raggedright{\textbf{Sport und Aktivitäten:} Ich genieße Aktivitäten wie Wandern, Badminton, Klettern, Motorradfahren, Skifahren und Fitness im Studio. Außerdem probiere ich gerne neue Sportarten aus.}
        \item \raggedright{\textbf{Networking und Events:} Ich nehme regelmäßig an Networking-Veranstaltungen im Raum Karlsruhe teil, darunter Start-Up Gründertreffen und verschiedene Meetups.}
        \item \raggedright{\textbf{Kulinarisches Experimentieren:} Ich liebe es, neue Rezepte auszuprobieren und mich kreativ auszuleben.} 
        \item \raggedright{\textbf{Kulturelle Erlebnisse:} Ich schätze kulturelle Veranstaltungen wie Konzerte und Festivals. Außerdem reise ich gerne und lerne so neue Menschen und Kulturen kennen.}
        \item \raggedright{\textbf{Fortbildung:} Mein Interesse an lebenslangem Lernen zeigt sich durch das Lesen von Fachliteratur und Romanen sowie das Anhören von Podcasts und TED-Talks.}
        \item \raggedright{\textbf{Sonstiges:} Schach, Brettspiele, Kino, soziales Engagement}
    \end{itemize}


\end{rSection}



\end{document}
